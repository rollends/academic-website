---
title :   "The Linear Algebra behind the Truncated Fourier Series"
author:   "Rollen S. D'Souza"
date  :   "2020-12-30"
---
Most, if not all, engineering and applied mathematics students are familiar with the Fourier series.
Here I will give a slightly different treatment of the Fourier series that I find is interesting.
It will have a similar flavour to that often seen in more formal mathematics classes that have covered it.
I will assume a fair knowledge of abstract linear algebra.

Define the set of functions
\[
  L_2([-\pi, \pi])
    =
      \left\{
        f: [-\pi, \pi] \to \mathbb{C}
        \;\middle|\;
        \int_{-\pi}^{\pi} f(t) \overline{f(t)} ~dt < \infty
      \right\},
\]
where we'll take \(f\) to also be piecewise continuous.
Functions in \(L_2([-\pi, \pi])\) are said to be \textbf{square integrable} on \([-\pi, \pi].\)
In engineering terminology, these are finite energy signals.
This space is an infinite-dimensional vector space over the field \(\mathbb{C}\) but it can be endowed with even stronger structure than this.
Given any two functions \(f,\) \(g\in L_2([-\pi, \pi])\) define their inner product by
\[
  \langle f, g \rangle_{L_2} = \int_{-\pi}^{\pi} f(t) \overline{g(t)}~dt.
\]
To avoid confusion, inner products over the space \(L_2([-\pi, \pi])\) will always be subscripted with \(L_2.\)
Observe that since all functions are square integrable we have that the induced norm
\[
  \|f\| = \sqrt{\langle f, f \rangle}
\]
is well-defined.

Now let \(\mathbb{C}^n\) have the usual inner product space structure.
Define the linear map \(\mathcal{L}: \mathbb{C}^n \to L_2([-\pi, \pi])\) by
\[
  \mathcal{L}(x)
    =
      x^1 + x^2 e^{i 2 \pi t} + \cdots + x^{n} e^{i 2 \pi (n-1) t},
\]
where \(x = (x^1, \ldots, x^n).\)
To this map, there exists an adjoint map \(\mathcal{L}^*: L_2([-\pi, \pi]) \to \mathbb{C}^n\) that satisfies, for all \(x \in \mathbb{C}^n\) and \(f \in L_2([-\pi, \pi]),\)
\[
  \langle \mathcal{L}(x), f \rangle_{L_2}
  =
  \langle x, \mathcal{L}^*(f) \rangle.
\]
Since the domain of \(\mathcal{L}\) is finite-dimensional it has a basis and so we can write both \(x\) and \(\mathcal{L}^*(f)\) in terms of this basis to find a formula for \(\mathcal{L}^*(f).\)
Using the standard orthonormal basis \(e_1, \ldots, e_n\) for \(\mathbb{C}^n,\)
\[
  \mathcal{L}^*(f)
    =
      \frac{\langle 1, f \rangle_{L_2}}{\langle 1, 1\rangle_{L_2}}
      e_1
      +
      \frac{\langle e^{i 2 \pi t}, f \rangle_{L_2}}{\langle e^{i 2 \pi t}, e^{i 2 \pi t}\rangle_{L_2}}
      e_2
      +
      \cdots
      +
      \frac{\langle e^{i 2 \pi (n-1) t}, f \rangle_{L_2}}{\langle e^{i 2 \pi(n-1)t}, e^{i 2 \pi(n-1) t}\rangle_{L_2}}
      e_n.
\]
Consider the following optimization problem.
%
\begin{quote}
  Given \(f \in L_2([-\pi, \pi]),\) what is the \(x^* \in \mathbb{C}^n\) that solves
  \[
    \min_{x \in \mathbb{C}^n} \langle L x - f, L x - f \rangle_{L_2}.
  \]
\end{quote}
%
This question asks to find the coefficients \(x\) to our complex exponentials that best approximates the function \(f \in L_2([-\pi, \pi]).\)
It turns out we can find an explicit expression for \(x^*.\)
Really, this problem is just a linear least squares problem on a finite-dimensional vector space.
Most students will have encountered this problem in a linear algebra class.
In some settings, the solution is directly computed by first computing the \href{https://en.wikipedia.org/wiki/Moore%E2%80%93Penrose_inverse}{Moore-Penrose pseudoinverse} of the matrix characterizing the problem.
We will take this approach.

Consider the map \(\mathcal{L}^* \circ \mathcal{L}: \mathbb{C}^n \to \mathbb{C}^n.\)
This is now a linear map between finite-dimensional vector spaces.
It is a very special kind of linear map.
Let \(x \in \mathbb{C}^n\) be arbitrary and consider the inner product
\[
  \langle x, (\mathcal{L}^* \circ \mathcal{L})(x) \rangle.
\]
Using the property of the adjoint the above expression equals
\[
  \langle \mathcal{L}(x), \mathcal{L}(x) \rangle.
\]
This is positive if, and only if, \(\mathcal{L}(x)\) is equal to zero function.
The only way for \(\mathcal{L}(x)\) to be the zero function is if \(x = 0\) since \(\mathcal{L}\) is injective by construction.
So it follows that \(\langle x, (\mathcal{L}^* \circ \mathcal{L})(x) \rangle > 0\) if, and only if, \(x \neq 0.\)
This is the definition of a map being positive definite so we conclude that \(\mathcal{L}^* \circ \mathcal{L}\) is positive definite.
Positive definite matrices are invertible.
We are now justified in defining
\[
  x^* = \left(\mathcal{L}^* \circ \mathcal{L}\right)^{-1}(\mathcal{L}^*(f)).
\]
Compare with your sources to see that \((\mathcal{L}^* \circ \mathcal{L})^{-1} \circ \mathcal{L}^*\) is really just the (left) pseudoinverse.
I claim that this \(x^*\) is the solution.

Skirting the proof here as it is a matter of second year linear algebra, what really is the value of this perspective?
For one, observe that nowhere did we have to use the fact that the complex exponentials form an orthogonal basis.
This is often a fact explicitly leveraged to demonstrate the formulas for the coefficients.
Two, note that there is no reason to rely on the standard inner product on \(\mathbb{C}^n.\)
Suppose we cared more about precisely estimating the lower frequencies than higher frequencies.
We could define an alternative inner product on \(\mathbb{C}^n\) that then determines a different map \(\mathcal{L}^*\) and possibly a different \(x^*\)!

Let us explore precisely this case with an example.
