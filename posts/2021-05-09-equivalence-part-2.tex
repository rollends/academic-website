---
title :   "`Equivalence' for Control Theorists: Part 2"
author:   "Rollen S. D'Souza"
date  :   "2021-05-09"
---
This is a continuation of ```Equivalence' for Control Theorists: Part 1.''
Now let us move to a problem of linear control theory.
Let \(x(t) \in \mathbb{R}^n\) be a state that evolves according to the linear control system
\begin{equation}
  \label{eqn:system}
  \dot{x}(t) = A\, x(t) + B\, u(t),
\end{equation}
where \(A \in \mathbb{R}^{n\times n},\) \(B \in \mathbb{R}^n\) and \(u(t) \in \mathbb{R}\) is the control input.
We would like to answer the following question.
%
\begin{quote}
  What conditions must the pair \((A, B)\) satisfy so that there exists an invertible linear transformation \(z = P\, x\) where (1) becomes, in the new coordinates,
  \begin{equation}
  \label{eqn:controllablesystem}
  \begin{aligned}
    \dot{z}_1(t) &= z_2(t),\\
    \dot{z}_2(t) &= z_3(t),\\
                 &\quad\vdots \\
    \dot{z}_n(t) &= u(t).
  \end{aligned}
  \end{equation}
  The system (2) is said to be in controllable canonical form.
\end{quote}
%
Recalling the philosophy from the previous post, we seek a unique property of the controllable canonical system (2) that is invariant under the transformation \(z = P x\);
here we are also working with an invertible linear transformation so we know that one useful invariant is the dimension of subspaces.

Compute the Laplace transform of (2) and rearrange to form the algebraic constraint
\[
  \begin{pmatrix}
    s      & -1     & 0      & \cdots & 0      & 0\\
    0      &  s     & -1     & \cdots & 0      & 0\\
    \vdots & \vdots & \vdots & \vdots & \vdots & \vdots\\
    0      &  0     & 0      & 0      & s      & -1\\
    0      &  0     & 0      & 0      & 0      & s
  \end{pmatrix}
  \begin{pmatrix}
    Z_1(s)\\
    Z_2(s)\\
    \vdots\\
    Z_{n-1}(s)\\
    Z_n(s)
  \end{pmatrix}
  -
  \begin{pmatrix} 0 \\ 0 \\ \vdots \\ 0 \\ 1 \end{pmatrix}
  U(s)
  =
  0.
\]
Collect this into one algebraic system
\[
  \left(
  \begin{array}{l|l}
    \begin{array}{cccccc}
      s      & -1     & 0      & \cdots & 0      & 0\\
      0      &  s     & -1     & \cdots & 0      & 0\\
      \vdots & \vdots & \vdots & \vdots & \vdots & \vdots\\
      0      &  0     & 0      & 0      & s      & -1\\
      0      &  0     & 0      & 0      & 0      & s
    \end{array}
    &
    \begin{array}{c} 0 \\ 0 \\ \vdots \\ 0 \\ -1 \end{array}
  \end{array}
  \right)
  \begin{pmatrix}
    Z_1(s)\\
    Z_2(s)\\
    \vdots\\
    Z_n(s)\\
    U(s)
  \end{pmatrix}
  =
  0.
\]
This system of \(n\) equations in \(n+1\) unknowns has rank \(n\) for all \(s\in \mathbb{C}.\)
This is an unusual property that not all systems have.
Not only that, it is a property that is preserved under the linear transformation.
Observe that for the system (1) we can compute the Laplace transform and arrive at a similar augmented system,
\[
  \left(
  \begin{array}{l|l}
    s\, I - A & -B
  \end{array}
  \right)
  \begin{pmatrix}
    X(s) \\ U(s)
  \end{pmatrix}
  =
  0.
\]
If we perform a coordinate change \(v = P x\) the equation becomes
\[
  P^{-1}
  \left(
  \begin{array}{l|l}
    s\, I - P A P^{-1} & -P B
  \end{array}
  \right)
  \begin{pmatrix}
    V(s) \\ U(s)
  \end{pmatrix}
  =
  0,
\]
where we observe that \(\bar{A} = P A P^{-1}\) and \(\bar{B} = P B\) are the new matrices of the state-space model for \(v(t).\)
As a result, we can see that the matrix's rank is left invariant by the invertible linear transformation.
Given a transformation \(z = P x\) we have
\[\begin{aligned}
  0
    &=
      P^{-1}
      \left(
      \begin{array}{l|l}
        s\, I - P A P^{-1} & -P B
      \end{array}
      \right)
      \begin{pmatrix}
        Z(s) \\ U(s)
      \end{pmatrix}\\
    &=
      P^{-1}
      \left(
      \begin{array}{l|l}
        \begin{array}{cccccc}
          s      & -1     & 0      & \cdots & 0      & 0\\
          0      &  s     & -1     & \cdots & 0      & 0\\
          \vdots & \vdots & \vdots & \vdots & \vdots & \vdots\\
          0      &  0     & 0      & 0      & s      & -1\\
          0      &  0     & 0      & 0      & 0      & s
        \end{array}
        &
        \begin{array}{c} 0 \\ 0 \\ \vdots \\ 0 \\ -1 \end{array}
      \end{array}
      \right)
      \begin{pmatrix}
        Z_1(s)\\
        Z_2(s)\\
        \vdots\\
        Z_n(s)\\
        U(s)
      \end{pmatrix}.
\end{aligned}\]
%
This process allows us to conjecture
%
\begin{quote}
  \textbf{Conjecture:}
  There exists a linear transformation \(z = P x\) so that system (1) is equivalent to system (2) if, and only if,
  \(
    \operatorname{rank}\left(
      \begin{array}{l|l}
        s I - A & B
      \end{array}
    \right)
    =
    n
  \)
  for all \(s \in \mathbb{C}.\)
\end{quote}
This conjecture turns out to be a theorem, the Popov-Belevitch-Hautus (PBH) Test, and the rank condition is one of the many equivalent conditions for controllability.
The PBH Test is seldom used, likely because the proof established in the literature is non-constructive;
it doesn't tell you how to construct \(P\) and, even if it did, it would probably be more complicated than other results.

