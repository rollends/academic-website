---
title :   "On the Heaviside Cover-up Method"
author:   "Rollen S. D'Souza"
date  :   "2021-02-15"
---
Engineering students take introductory courses in ordinary differential equations where they encounter a barrage of Laplace transforms they must invert.
The formal inverse is a contour integral over the region of convergence of the transform in the \(s\)-domain but this is difficult to do efficiently.
Since engineering courses primarily discuss finite-dimensional linear ordinary differential equations there is a specific, restricted class of transforms they observe.
All the Laplace transforms an engineer will see take the form
\[
  F(s) = \frac{a_{n-1} s^{n-1} + \cdots + a_1 s + a_0}{s^n + b_{n-1} s^{n-1} + \cdots + b_1 s + b_0}
\]
where \(a_i, b_i \in \mathbb{R}.\)
The fundamental theorem of algebra allows us to rewrite the denominator as
\[
  F(s) = \frac{a_{n-1} s^{n-1} + \cdots + a_1 s + a_0}{\prod (s - r_i)^{\kappa_i}}
\]
where \(r_i \in \mathbb{C}\) are the unique roots of the polynomial with multiplicity \(\kappa_i \in \mathbb{N}.\)
Observe that the right hand-side, by algebra, can be written as a finite sum of rational maps
\[
  F(s) = \sum_i \sum_{k = 1}^{\kappa_i} \frac{A_{i, k}}{(s - r_i)^k}
\]
It is then a matter of collapsing this summation into one fraction and matching the unknown coefficients \(A_{i, k}\) with those that form the numerator polynomial \(a_{n-1},\ldots, a_0\); this results in a \emph{large} system of linear equations.
Solving a linear system of equations is how a computer would do it.
But how would a practicing engineer do it without a computer?

\section{Heaviside Cover-up Method}
The \href{https://en.wikipedia.org/wiki/Heaviside_cover-up_method}{Heaviside Cover-up Method} is a method named after engineer Oliver Heaviside to discover the coefficients \(A_{i, k}\) through direct evaluation.
Fix a root \(r_i.\)
There are \(\kappa_i\) coefficients \(A_{i, 1}\) up to \(A_{i, \kappa_i}\) associated to this root.
Observe that if we multiply both sides of the partial fraction expansion by \((s - r_i)^{\kappa_i}\) we arrive at
\[
  (s - r_i)^{\kappa_i} F(s)
    =
    A_{i, \kappa_i}
    +
    \sum_{k=1}^{\kappa_i - 1} (s - r_i)^{\kappa_i - k} A_{i, k}
    +
    \sum_{j\neq i} \sum_{k = 1}^{\kappa_i} A_{i, k} \frac{(s - r_i)^{\kappa_i}}{(s - r_i)^k}.
\]
Now consider the limit of both sides of this expression as \(s\to r_i.\)
Necessarily the latter two summations must vanish, as they have zeros at \(r_i\) but no poles (infinite discontinuities at \(r_i\)).
This results in the rule
\[
  A_{i, \kappa_i} = \lim_{s \to r_i} (s - r_i)^{\kappa_i} F(s).
\]
In simpler terms, to produce the coefficient \(A_{i, \kappa_i}\) evaluate \(F(s)\) at \(r_i\) but exclude the denominator factors of \(F(s)\) that will blowup at \(r_i.\)
This is all pretty well explained in the Wikipedia article linked.
However there is a piece that is left out of the Wikipedia article as it stands.
How do you solve for all the other coefficients associated to the lower powers of \(r_i\)?
Wikipedia gives an answer that puts me off, quite frankly:
\begin{quote}
  Since the equation of the numerators... is true for all values of \(x,\) pick a value for \(x\) and use it to solve for [\(A_{i, \kappa_i - 1}\)].
\end{quote}
This is obviously absurd. 
It works for second order roots, but what about if the root was repeated 5 times over?
You will need to evaluate at 4 points!
Moreover, it begs the question: why bother with heaviside cover-up and just use this point evaluation technique on the original \(F(s)\) at points away from its discontinuities.
That too determines the coefficients of the partial fraction expansion!

\textbf{No.}
We will not take this approach.
Because it is actually far simpler than this.
Since we know the coefficient \(A_{i, \kappa_i}\) we can define a \emph{new} Laplace transform
\[
  G(s)
    = 
      F(s) - \frac{A_{i, \kappa_i}}{(s - r_i)^{\kappa_i}}.
\]
Facts:
\begin{enumerate}
  \item{\(F(s)\) has a partial fraction expansion with root \(r_i\) appearing with order \(\kappa_i\),}
  \item{the latter term cancels this higher order term in the partial fraction expansion of \(F(s),\)}
  \item{\(G(s)\) is now expressed as the partial fraction expansion of \(F(s)\) with root \(r_i\) appearing at an order less than, or equal to, \(\kappa_i - 1,\)}
  \item{\(G(s)\) has a pole at \(r_i\) with order \emph{at most}, but possibly less than, \(\kappa_i - 1,\)}
  \item{the coefficients of the partial fraction expansion of \(G(s)\) are equal to that of \(F(s)\) with the exception of the \(A_{i, \kappa_i}\) that does not appear.}
\end{enumerate}
So now we can repeat the algorithm described above for \(G(s)\) to solve for the lower order coefficients.
In particular,
\[
  A_{i, \kappa_i - 1}
    =
      \lim_{s \to r_i} (s - r_i)^{\kappa_i - 1} G(s)
    =
      \lim_{s \to r_i} \frac{(s - r_i)^{\kappa_i} F(s) - A_{i, \kappa_i}}{s - r_i}.
\]
Repeat until all the coefficients associated to the root \(r_i\) are resolved.
And yes, since the other coefficients remain unmutated by this process of producing new transforms, you can use the \emph{reduced order} transform to compute the other coefficients as well thereby simplifying the computations you perform at every step!

\section{Example}
It is best to demonstrate by way of example.
Consider the transform
\[
  F(s)
    =
      \frac{s - 3}{(s + 3)^3 (s - 1)}.
\]
We know we can write this function in expanded form as
\[
  F(s)
    =
      \frac{A_{1,1}}{s + 3} + \frac{A_{1,2}}{(s + 3)^2} + \frac{A_{1,3}}{(s + 3)^3}
      +
      \frac{A_{2,1}}{s - 1}. 
\]
Coefficient \(A_{1, 3}\) is, by our formula,
\[
  A_{1, 3} 
    =
      \lim_{s \to -3} (s + 3)^3 F(s)
    =
      \lim_{s \to -3} \frac{s - 3}{s - 1}
    =
      \frac{3}{2}.
\]
Substituting this number into our partial fraction expansion
\[
  F(s)
    =
      \frac{A_{1,1}}{s + 3} + \frac{A_{1,2}}{(s + 3)^2} + \frac{3/2}{(s + 3)^3}
      +
      \frac{A_{2,1}}{s - 1}. 
\]
Define \(G(s) = F(s) - (3/2)1/(s + 3)^3.\)
Simplify this to obtain an expression for \(G(s),\)
\[
  G(s)
    =
      \frac{s - 3}{(s + 3)^3 (s - 1)} - \frac{(3/2)(s - 1)}{(s + 3)^3 (s - 1)}
    =
      \frac{-\frac{1}{2}s - \frac{3}{2}}{(s + 3)^3 (s - 1)}
    =
      -\frac{1}{2}
      \frac{1}{(s+3)^2 (s - 1)}.
\]
Moreover, from our partial fraction expansion for \(F(s)\) we see that \(G(s)\) must also take the form
\[
  G(s)
    =
      \frac{A_{1,1}}{s + 3} + \frac{A_{1,2}}{(s + 3)^2}
      +
      \frac{A_{2,1}}{s - 1},
\]
where the \(A_{i, k}\) are the same variables as before.
As a result, we can compute \(A_{1, 2}\) using our formula again
\[
  A_{1, 2} 
    =
      \lim_{s \to -3} (s + 3)^2 G(s)
    =
      \lim_{s \to -3} -\frac{1}{2}\frac{1}{s - 1}
    =
      \frac{1}{8}.
\]
So we have that
\[
  G(s)
    =
      \frac{A_{1,1}}{s + 3} + \frac{1/8}{(s + 3)^2}
      +
      \frac{A_{2,1}}{s - 1}.
\]
Once again, define a new function \(H(s) = G(s) - (1/8) 1/(s+3)^2\) and simplify to arrive at the expression
\[
\begin{aligned}
  H(s) 
    &=
      -\frac{1}{2}
      \frac{1}{(s+3)^2 (s - 1)}
      -
      \frac{1}{8}
      \frac{s - 1}{(s+3)^2 (s-1)}\\
    &=
      -\frac{1}{2}
      \left[
        \frac{\frac{1}{4} s + \frac{3}{4}}{(s+3)^2 (s-1)}
      \right]\\
    &=
      -\frac{1}{8}
      \frac{1}{(s+3)(s-1)}.
\end{aligned}
\] 
The partial fraction expansion of \(H(s)\) must take the form
\[
  H(s)
    =
      \frac{A_{1,1}}{s + 3}
      +
      \frac{A_{2,1}}{s - 1}.
\]
At this point it should be clear the advantages of employing this strategy.
Of course these final coefficients can be found by employing the same formulas:
\[
  A_{1, 1} = \frac{1}{32},\quad A_{2,1} = -\frac{1}{32}.
\]
This gives the complete partial fraction expansion for \(F(s)\)
\[
  F(s)
    =
      \frac{1/32}{s + 3} + \frac{1/8}{(s + 3)^2} + \frac{3/2}{(s + 3)^3}
      +
      \frac{-1/32}{s - 1}. 
\]
