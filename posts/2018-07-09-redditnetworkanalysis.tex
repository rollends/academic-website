---
title :   "Applying Network Science to Reddit: Introduction"
author:   "Rollen S. D'Souza"
date  :   "2018-07-09"
---
Although my research tends to stay in the \(C^\infty\) domain, I decided this term to take a course on Networked and Distributed systems.
The course takes a dip into applications of dynamical systems on nodes of a graph, which is a marked departure from my usual comfort zone.
In this spirit, I decided to take an even farther trip away from my domain of control into the forest of network science for the purpose of my course project.
I intend to track my thoughts in a series of blog posts;
I hope they serve as an interesting read as well as a method for me to keep my head straight.
The intention of my project is to use a variety of tools in network science to detect interesting users or posts on communities (subreddits) on Reddit.
In particular, my current intentions are to:
%
\begin{itemize}
 \item{Detect key posts, users or comments of a community in a given time period.}
 \item{Detect sub-clusters within a community. Who talks to who?}
\end{itemize}
%
The intent is to analyze the networks I am most familiar with.
These are
%
\begin{itemize}
  \item{\texttt{/r/science/}}
  \item{\texttt{/r/uwaterloo/}}
  \item{\texttt{/r/badmathematics/}}
\end{itemize}
%

\paragraph{Has this been done before?}
As far as I could find, not many have performed this exact of analysis in the literature.
This is certainly due to the fact that it is a fairly specific question to ask that doesn't yield any high level abstract information about Reddit itself.
In fact, it yields information that is specific to that community that is analyzed and may reveal information about how specific users interact in the community.
Not a very good research question.
But a great applied project.

A number of authors in the literature were interested in how information propogates via Reddit into other information vessels~\cite{c1} or how these networks influence behaviour.
This is naturally where the most interesting questions, but requires a careful study design that I did not clearly take time to do.
\emph{Hey, I'm lazy.}
Nonetheless, the excellent work by the authors of~\cite{c2} and~\cite{c3} inspires the graph structure I will use to analyze specific influencing behaviour of posts, comments and users.

In terms of "community" analyses,~\cite{c4} and~\cite{c5} are quite typical of the type of analyses one sees.
They are mostly interested either on how communities evolve or how information propogates through these communities.
For what its worth, these are amazingly technical and make for a good read.
Instead, I will apply the techniques of community detection to find groups of users that interact tightly with one another.
There a variety of fancy techniques available here, but I will stick to a generalization of a very simple technique discussed in class.
The structure of the posts following this will be to introduce the necessary mathematical formalisms and dive into the model and analysis.

\paragraph{The Implementation}
I have chosen to make my code and report public on my \href{https://github.com/rollends/red-prism}{Github profile}). The project name is Red Prism.
\href{https://en.wikipedia.org/wiki/PRISM_(surveillance_program)}{The irony of this name is not to be left unobserved}.
Don't worry, no specific information about specific users will be placed in the report.

The scripts will be written in Python 3.
It will use a Redis (key-value database) store to temporarily store information on posts, comments and user names.
It is at this point I want to make a great, big shout-out to `/u/Stuck\_In\_The\_Matrix` for putting together archives of Reddit data and uploading it to~\cite{c6}.
This is incredibly valuable and I plan to make a donation to the server upon completion of this project.
The scripts rely on the whole archives, filtering through them and storing the filtered data into the Redis store.
This is mostly to prevent unnecessary network traffic during analysis;
that is, to allow me to easily transform data into any format I like without having to redownload anything.

\paragraph{Conclusion}
This is a fairly small project in terms of the level of algorithms and mathematical tools involved.
However I hope even with the limited technical tools I am trained with, I hope to find some interesting information about the communities I care about!
And maybe you'll learn something cool too!

\paragraph{References}
\bibitem{c1}: Haralabopoulos et al., ``Lifespan and propagation of information in Online Social Networks: A case study based on Reddit,'' \emph{Journal of Network and Computer Applications}, doi: \href{https://doi.org/10.1016/j.jnca.2015.06.006}{10.1016/j.jnca.2015.06.006}.

\bibitem{c2}: Tsantarliotis P. et al., ``Defining and predicting troll vulnerability in online social media,'' \emph{Social Network Analysis and Mining}, vol. 7, no. 26, December 2017, doi: \href{https://doi.org/10.1007/s13278-017-0445-2}{10.1007/s13278-017-0445-2}.

\bibitem{c3}: Medvedev A. et al., ``Modelling structure and predicting dynamics of discussion threads in online boards,'' arxiv: \href{https://arxiv.org/pdf/1801.10082.pdf}{1801.10082}.

\bibitem{c4}: Zannettou S. et al., "On the Origins of Memes by Means of Fringe Web Communities," arxiv: [1805.12512](https://arxiv.org/pdf/1805.12512.pdf).

\bibitem{c5}: Kumar S. et al., ``Community Interaction and Conflict on the Web,'' arxiv: \href{https://arxiv.org/pdf/1803.03697.pdf}{1803.03697}.

\bibitem{c6}: \href{https://files.pushshift.io/reddit/}{Reddit PushShift Archive}
